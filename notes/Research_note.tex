\documentclass[a4paper,11pt,dvipdfmx]{jsarticle}
\usepackage{amsmath, amssymb, amsthm}
\usepackage{bm}
\usepackage{mathtools}


% --- 定理・定義環境 ---
\theoremstyle{definition}
\newtheorem{definition}{定義}[section]
\newtheorem{theorem}[definition]{定理}
\newtheorem{lemma}[definition]{補題}
\newtheorem{assumption}{仮定}
\newtheorem{proposition}[definition]{命題}
\newcommand{\aU}{\underline{\bm{a}}}
\newcommand{\bU}{\underline{\bm{b}}} 
\newcommand{\cU}{\underline{\bm{c}}} 
\newcommand{\dU}{\underline{\bm{d}}}
\newcommand{\eU}{\underline{\bm{e}}}
\newcommand{\fU}{\underline{\bm{f}}}
\newcommand{\gU}{\underline{\bm{g}}}
\newcommand{\hU}{\underline{\bm{h}}}
\newcommand{\iU}{\underline{\bm{i}}}
\newcommand{\jU}{\underline{\bm{j}}}
\newcommand{\kU}{\underline{\bm{k}}}
\newcommand{\lU}{\underline{\bm{l}}}
\newcommand{\mU}{\underline{\bm{m}}}
\newcommand{\nU}{\underline{\bm{n}}}
\newcommand{\oU}{\underline{\bm{o}}}
\newcommand{\pU}{\underline{\bm{p}}}
\newcommand{\qU}{\underline{\bm{q}}}
\newcommand{\rU}{\underline{\bm{r}}}
\newcommand{\sU}{\underline{\bm{s}}}
\newcommand{\tU}{\underline{\bm{t}}}
\newcommand{\uU}{\underline{\bm{u}}}
\newcommand{\vU}{\underline{\bm{v}}}
\newcommand{\wU}{\underline{\bm{w}}}
\newcommand{\xU}{\underline{\bm{x}}}
\newcommand{\yU}{\underline{\bm{y}}}
\newcommand{\zU}{\underline{\bm{z}}}

\newcommand{\zeroU}{\underline{0}}
\newcommand{\C}[2]{C_{{#1},{#2}}}
\newcommand{\calC}{\mathcal{C}}
\newcommand{\cy}[3]{\mathcal{C}_{{#1},{#2},{#3}}}
\newcommand{\inv}{^{-1}}
\newcommand{\HHX}{\hat{H}_X}
\newcommand{\HHZ}{\hat{H}_Z}
\newcommand{\HX}{H_X}
\newcommand{\HZ}{H_Z}
\newcommand{\tHX}{\tilde{H}_X}
\newcommand{\tHZ}{\tilde{H}_Z}
\newcommand{\tr}{^\top}
\newcommand{\Row}{\text{Row}}
\newcommand{\wt}{\text{wt}}

\newcommand{\ACfour}[5]{A_{#1} &= a_{#2}\inv a_{#3} a_{#4}\inv a_{#5}}
\newcommand{\BCfour}[5]{B_{#1} &= a_{#2}\inv(b_{#3} - b_{#2}) + a_{#2}\inv a_{#3} a_{#4}\inv (b_{#5} - b_{#4})}
\newcommand{\ABCfour}[5]{\ACfour{#1}{#2}{#3}{#4}{#5}\\\BCfour{#1}{#2}{#3}{#4}{#5}}


% =========================================
\begin{document}

\title{\textbf{研究ノート}}
\author{黒澤雅治}
\date{2025/12/21}
\maketitle

\section{$\fU, \gU$の条件を満たす解空間の特定}
\subsection{可換性条件}
$f=a_1x+b_1,g=a_2x+b_2$とする。$(a_1,b_1,a_2,b_2\in[P], a_1, a_2\neq 0)$
可換性条件は、すべての$x\in[P]$に対して
\begin{align}
  &f(g(x)) = g(f(x))\\
  \Longleftrightarrow &a_1(a_2x+b_2)+b_1 = a_2(a_1x+b_1)+b_2\\
  \Longleftrightarrow &a_1a_2x + a_1b_2 + b_1 = a_2a_1x + a_2b_1 + b_2\\
  \Longleftrightarrow &a_1b_2 + b_1 = a_2b_1 + b_2
\end{align}

\subsection{サイクル}
サイクルは、$L$列から偶数個の列を選び、開始行を指定することで一意に特定される。
以下、$L=6, J=2$に固定して考える。

\subsubsection{同一とみなせるサイクル}
$\ell$のサイクルは、$\ell / 2$個の列から構成される。
よって、長さ$2L$以下のサイクルを考慮するには、$2,4,\dots,L$個の列からなるサイクルを考えればよい。
\begin{definition}
  列シーケンス$\ell$により構成され、開始位置が$j$行目であるサイクルを$\C{\ell}{j}$で表し、サイクルは行列における(0から始まる)インデックスの列として表現する。
\end{definition}

$\ell=[a,b,c,d]$について考える。
このとき、
\begin{align}
  \C{\ell}{0}=([0,a],[0,b],[1,b],[1,c],[0,c],[0,d],[1,d],[1,a])\\
  \C{\ell}{1}=([1,a],[1,b],[0,b],[0,c],[1,c],[1,d],[0,d],[0,a])
\end{align}

\begin{definition}
  1つの左シフトを$S(\cdot)$で表現し、$i$個左シフトする操作は$S^i(\cdot)$で表す。右シフトは$S^{-1}(\cdot)$で表す。
\end{definition}
1つシフトしたもの、2つシフトしたものについて考える(それ以上のシフトはこれらの組み合わせで再現できる。)
$\ell=[a,b,c,d]$を1つシフトした$S(\ell)=[b,c,d,a]$について、
\begin{align}
  \C{S(\ell)}{0}=([0,b],[0,c],[1,c],[1,d],[0,d],[0,a],[1,a],[1,b])=\C{\ell}{1}\\
  \C{S(\ell)}{1}=([1,b],[1,c],[0,c],[0,d],[1,d],[1,a],[0,a],[0,b])=\C{\ell}{0}
\end{align}
$\ell=[a,b,c,d]$を2つシフトした$S^2(\ell)=[c,d,a,b]$について、
\begin{align}
  \C{S^2(\ell)}{0}=([0,c],[0,d],[1,d],[1,a],[0,a],[0,b],[1,b],[1,c])=\C{\ell}{0}\\
  \C{S^2(\ell)}{1}=([1,c],[1,d],[0,d],[0,a],[1,a],[1,b],[0,b],[0,c])=\C{\ell}{1}
\end{align}
以上から、以下の定理が成り立つ。
\begin{theorem}
  \begin{align}
    \C{(\ell)}{0} = \C{(S(\ell))}{1} = \C{S^2(\ell)}{0}\\
    \C{(\ell)}{1} = \C{(S(\ell))}{0} = \C{S^2(\ell)}{1}\\
  \end{align}
\end{theorem}
よって、ある列シーケンス$\ell$について、開始行が0,1行目である2つのサイクルを考えれば、列シーケンスをシフトしたものから構成されるサイクルすべてを網羅できる。
% \paragraph{反転}
% $\ell=[a,b,c,d]$を反転した$\ell'=[d,c,b,a]$について、
% \begin{align}
%   \C{\ell'}{0}=([0,d],[0,c],[1,c],[1,b],[0,b],[0,a],[1,a],[1,d])\\
%   \C{\ell'}{1}=([1,d],[1,c],[0,c],[0,b],[1,b],[1,a],[0,a],[0,d])
% \end{align}

\end{document}