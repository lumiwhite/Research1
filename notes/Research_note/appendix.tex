
\begin{appendix}
\section{定理\ref{th:numberOfSequence}の証明}\label{app:proof1}
\begin{proof}
  まず、円環状にする前の、長さ$n$の直線の列$x_1, x_2, \dots, x_n$を考える。隣り合う要素$x_i, x_{i+1}$は常に異なるとする。
  この直線の列全体における色の塗り分けの総数は、先頭$x_1$が$k$通り、それ以降の$x_2, \dots, x_n$がそれぞれ直前の要素と異なるため$k-1$通りであることから、
  \begin{align}
    W_n = k(k-1)^{n-1}
  \end{align}
  である。

  ここで、$W_n$を以下の2つのケースに分割する。
  \begin{itemize}
    \item $a_n$:始点と終点が同じ色である場合の数 ($x_1 = x_n$)
    \item $b_n$:始点と終点が異なる色である場合の数 ($x_1 \neq x_n$)
  \end{itemize}
  すなわち、
  \begin{align}
    a_n + b_n = k(k-1)^{n-1} \label{eq:total}
  \end{align}
  である。円環状に接続した際に条件を満たすのは、$x_1 \neq x_n$ の場合であるため、求める値は$b_L$となる。

  次に、$n$から$n+1$への漸化式を考える。長さ$n$の列に、条件を満たすように新たな要素$x_{n+1}$を追加する。
  \begin{enumerate}
    \item $x_1 = x_{n+1}$ となる場合($a_{n+1}$を構成する場合):
    $x_{n+1}$は$x_n$と異なる必要がある。また、$x_{n+1}$は$x_1$と同じ色になるため、必然的に$x_n \neq x_1$でなければならない。
    つまり、$x_1 \neq x_n$である状態($b_n$通り)の末尾に、$x_1$と同じ色(1通り)を追加する場合のみ発生する。
    \begin{align}
      a_{n+1} = b_n \times 1 = b_n \label{eq:rec_a}
    \end{align}
    
    \item $x_1 \neq x_{n+1}$ となる場合($b_{n+1}$を構成する場合):
    これは全事象から$a_{n+1}$を引いたものである。式(\eqref{eq:total})より、
    \begin{align}
      b_{n+1} = k(k-1)^n - a_{n+1}
    \end{align}
  \end{enumerate}

  式(\eqref{eq:rec_a})を代入すると、以下の$b_n$に関する漸化式が得られる。
  \begin{align}
    b_{n+1} = k(k-1)^n - b_n
  \end{align}
  この漸化式を解く。
  \begin{align}
    b_{n+1} - (k-1)^{n+1} &= -(b_n - (k-1)^n) \\
    b_n - (k-1)^n &= (-1)^{n-2} (b_2 - (k-1)^2)
  \end{align}
  ここで、$n=2$の場合、隣り合う要素は異なるため必ず$x_1 \neq x_2$となる。よって$a_2 = 0, b_2 = k(k-1)$である。
  \begin{align}
    b_2 - (k-1)^2 &= k(k-1) - (k-1)^2 \\
    &= (k-1)(k - (k-1)) \\
    &= k-1
  \end{align}
  したがって、
  \begin{align}
    b_n - (k-1)^n &= (-1)^{n-2}(k-1) \\
    b_n &= (k-1)^n + (-1)^n(k-1)
  \end{align}
  以上より、長さ$L$の円環状の列シーケンスの総数は $(k-1)^L + (-1)^L(k-1)$ となる。
\end{proof}

\section{\ref{ssec:analyzeCondA}の解の導出}

\paragraph{$a_f\neq1$かつ$a_g \neq1$のとき\\}
  $\gcd(a_f-1, P)=1$のとき、$(a_f-1)\inv$が存在するので、$b_g =(a_f-1)\inv(a_g -1)b_f$により解が一意に決まる。

  $\gcd(a_f-1, P)=d>1$のとき、
  \begin{align}
      A &= a_f - 1 \\
      C &= (a_g  - 1)b_f \\
      x &= b_g 
  \end{align}とする。
  これにより、式(\eqref{eq:original})は以下の1次合同方程式となる。
  \begin{equation}
      Ax \equiv C \pmod P \label{eq:simplified}
  \end{equation}

  合同式の定義より、ある整数 $k$ が存在して $Ax - C = Pk$ が成り立つ。
  これを変形すると、以下の1次不定方程式が得られる。
  \begin{equation}
      Ax - Pk = C \label{eq:diophantine}
  \end{equation}
  $A$ と $P$ はともに $d$ の倍数であるため、互いに素な整数 $A', P'$を用いて以下のように表せる。
  \begin{equation}
      A = d A', \quad P = d P'
  \end{equation}
  これらを式(\eqref{eq:diophantine})に代入すると、
  \begin{align}
      d A' x - d P' k &= C \\
      d(A' x - P' k) &= C \label{eq:factorized}
  \end{align}
  左辺は整数 $d$ と整数の積であるため $d$ の倍数である。したがって、等式が成立するためには、右辺 $C$ も $d$ で割り切れなければならない。
  これより、以下の2つのケースに分類される。
  \begin{description}
    \item[$C \not\equiv 0 \pmod d$\\]
      等式を満たす整数 $x, k$ は存在しない。
      したがって、この場合、元の合同方程式の解は存在しない。
    \item[$C \equiv 0 \pmod d$\\]
    $C = d C'$ となる整数 $C'$ が存在する。
    式(\eqref{eq:factorized})の両辺を $d$ で割ると、
    \begin{equation}
      A' x - P' k = C' \label{eq:reduced}
    \end{equation}
    となる。これを合同式に戻すと、法 $P'$ における方程式が得られる。
    \begin{equation}
      A' x \equiv C' \pmod{P'} \label{eq:reduced_mod}
    \end{equation}
    ここで $\gcd(A', P') = 1$ であるため、法 $P'$ において $A'$ の逆元 $(A')^{-1}$ が一意に存在する。
    よって、式(\eqref{eq:reduced_mod})は法 $P'$ においてただ1つの解 $x_0$ を持つ。
    \begin{equation*}
      x \equiv x_0 \pmod{P'} \implies x = x_0 + m P' \quad (m \in \mathbb{Z})
    \end{equation*}
    元の法 $P$ ($= d P'$) における解 $x$ を求めるため、$0 \le x < P$ の範囲にある解を探す。
    \begin{align*}
        0 &\le x_0 + m P' < d P' \\
        0 &\le \frac{x_0}{P'} + m < d
    \end{align*}
    $x_0$ を $0 \le x_0 < P'$ と選べば、これを満たす整数 $m$ は $0, 1, \dots, d-1$ の計 $d$ 個存在する。
    したがって、解は$b_g  = x_0+nP'(n=0,1,\dots,d-1)$
  \end{description}

\paragraph{$a_f=1$かつ$a_g \neq1$のとき}
$A=0$より左辺が0になるので、右辺が0、すなわち$C=0$であれば任意の$b_g $が解となり、そうでない場合は解なしとなる。

\paragraph{$a_g =1$のとき\\}
このとき、$a_g -1=0$より$C=0$である。よって、右辺の$Ax$が$P$の倍数であればよい。\\
$a_f=1$の時、$A=0$より$Ax=0$となるため、任意の$b_g $で式\eqref{eq:original}は成り立つ。\\
$a_f\neq 1$のとき、以下の2つの場合に分けて考える。
\begin{description}
  \item[$\gcd(A, P)=1$]
    解は$b_g =0$のみ
  \item[$\gcd(A, P)=d>1$]
    $A = dA', P = dP'$とすると、$x$は$dA'x=mP=mdP'$を満たせばよい($m\in\mathbb{Z}$)。
    $A'$と$P'$は互いに素なので、$x$は$P'$の倍数であればよい。
    よって、解は$b_g =nP'(n=0,1,\dots,d-1)$となる。
\end{description}
\end{appendix}
